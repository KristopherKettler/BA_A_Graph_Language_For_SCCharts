% header.tex
\documentclass[a4paper,11pt,twoside,ngerman,color]{book}
\usepackage[a4paper,left=3.5cm,right=2.5cm,bottom=3.5cm,top=3cm]{geometry}

\usepackage[german,english]{babel}

\usepackage[pdftex]{graphicx,color}
\usepackage{amsmath,amssymb,subfigure}

% Theorem-Umgebungen
\usepackage[amsmath,thmmarks]{ntheorem}

% Korrekte Darstellung der Umlaute
\usepackage[utf8]{inputenc}
\usepackage[T1]{fontenc}

% Algorithmen
\usepackage[plain,chapter]{algorithm}
\usepackage{algorithmic}

\usepackage{enumerate}

\usepackage{acronym}
\usepackage{listings}
\usepackage{color}
\lstdefinelanguage{Xtend}{
	morekeywords={cached,case,default,extension,false,JAVA,WORKFLOWSLOT,let,new,null,private,create,switch,this,true,reexport,around,if,then,else, def, val, var, private, class, static, return, as, instanceof, for, override, boolean,ENDFOR,FOR,ENDIF,IF},
	morekeywords=[3]{FAST, NORMAL, IN, PROVIDED},
	keywordstyle=[3]\color{eclipseBlue}\textsl,
	morekeywords=[4]{genDeclaration,genSuspension,genAction,genRegion,file,head,find},                        % <-- Your keywords here (w/ orange)
	keywordstyle=[4]\color{eclipseOrange},
    deletekeywords={scchart.},
	morecomment=[l]{//}, 
	morecomment=[s]{/*}{*/}, 
}

\definecolor{eclipseOrange}{RGB}{200,48,0}
\definecolor{eclipseBlue}{RGB}{0,0,172}
\definecolor{mygreen}{rgb}{0,0.6,0}
\definecolor{mygray}{rgb}{0.5,0.5,0.5}
\definecolor{mymauve}{rgb}{0.58,0,0.82}
\definecolor{compiler}{rgb}{0.53,0.12,0.47}
\lstset{ 
  backgroundcolor=\color{white},   % choose the background color; you must add \usepackage{color} or \usepackage{xcolor}; should come as last argument
  basicstyle=\footnotesize,        % the size of the fonts that are used for the code
  breakatwhitespace=false,         % sets if automatic breaks should only happen at whitespace
  breaklines=true,                 % sets automatic line breaking
  morecomment=[s][\color{red}]{/∗ }{ ∗/},
  captionpos=b,                    % sets the caption-position to bottom
  commentstyle=\color{mygreen},    % comment style
  deletekeywords={delete},            % if you want to delete keywords from the given language
  escapeinside={\%*}{*)},          % if you want to add LaTeX within your code
  extendedchars=true,              % lets you use non-ASCII characters; for 8-bits encodings only, does not work with UTF-8
  %firstnumber=1000,                % start line enumeration with line 1000
  frame=single,	                   % adds a frame around the code
  keepspaces=true,                 % keeps spaces in text, useful for keeping indentation of code (possibly needs columns=flexible)
  keywordstyle=\color{compiler},       % keyword style
  language=Octave,                 % the language of the code
  morekeywords={*,id,graphModel,container,node,edge,diagramExtension,style,containableElements,attr,extends,prime,this,EString,EBoolean,as,incomingEdges,outgoingEdges,position,appearance,lineWidth,lineStyle,background,foreground,font,BOLD,nodeStyle,roundedRectangle,size,corner,text,CENTER,TOP,value,rectangle,LEFT,MIDDLE,ellipse,edgeStyle,location,ARROW,movable,CIRCLE,decorator,points,appearanceProvider,polyline,polygon},            % if you want to add more keywords to the set
  morecomment=[s]{/*}{*/},
  morecomment=[l]{//},
  deletekeywords={delete,info},
  numbers=left,                    % where to put the line-numbers; possible values are (none, left, right)
  numbersep=5pt,                   % how far the line-numbers are from the code
  numberstyle=\tiny\color{mygray}, % the style that is used for the line-numbers
  rulecolor=\color{black},         % if not set, the frame-color may be changed on line-breaks within not-black text (e.g. comments (green here))
  showspaces=false,                % show spaces everywhere adding particular underscores; it overrides 'showstringspaces'
  showstringspaces=false,          % underline spaces within strings only
  showtabs=false,                  % show tabs within strings adding particular underscores
  stepnumber=2,                    % the step between two line-numbers. If it's 1, each line will be numbered
  stringstyle=\color{blue},     % string literal style
  tabsize=2,	                   % sets default tabsize to 2 spaces
  title=\lstname                   % show the filename of files included with \lstinputlisting; also try caption instead of title
}

% Bibtex deutsch
\usepackage{bibgerm}

% URLs
\usepackage{url}

% Caption Packet
\usepackage[margin=0pt,font=small,labelfont=bf]{caption}
% Gliederung einstellen
%\setcounter{secnumdepth}{5}
%\setcounter{tocdepth}{5}

% Theorem-Optionen %
\theoremseparator{.}
\theoremstyle{change}
\newtheorem{theorem}{Theorem}[section]
\newtheorem{satz}[theorem]{Satz}
\newtheorem{lemma}[theorem]{Lemma}
\newtheorem{korollar}[theorem]{Korollar}
\newtheorem{proposition}[theorem]{Proposition}
% Ohne Numerierung
\theoremstyle{nonumberplain}
\renewtheorem{theorem*}{Theorem}
\renewtheorem{satz*}{Satz}
\renewtheorem{lemma*}{Lemma}
\renewtheorem{korollar*}{Korollar}
\renewtheorem{proposition*}{Proposition}
% Definitionen mit \upshape
\theorembodyfont{\upshape}
\theoremstyle{change}
\newtheorem{definition}[theorem]{Definition}
\theoremstyle{nonumberplain}
\renewtheorem{definition*}{Definition}
% Kursive Schrift
\theoremheaderfont{\itshape}
\newtheorem{notation}{Notation}
\newtheorem{konvention}{Konvention}
\newtheorem{bezeichnung}{Bezeichnung}
\theoremsymbol{\ensuremath{\Box}}
\newtheorem{beweis}{Beweis}
\theoremsymbol{}
\theoremstyle{change}
\theoremheaderfont{\bfseries}
\newtheorem{bemerkung}[theorem]{Bemerkung}
\newtheorem{beobachtung}[theorem]{Beobachtung}
\newtheorem{beispiel}[theorem]{Beispiel}
\newtheorem{problem}{Problem}
\theoremstyle{nonumberplain}
\renewtheorem{bemerkung*}{Bemerkung}
\renewtheorem{beispiel*}{Beispiel}
\renewtheorem{problem*}{Problem}

% Algorithmen anpassen %
\renewcommand{\algorithmicrequire}{\textit{Eingabe:}}
\renewcommand{\algorithmicensure}{\textit{Ausgabe:}}
\floatname{algorithm}{Algorithmus}
\renewcommand{\listalgorithmname}{Algorithmenverzeichnis}
\renewcommand{\algorithmiccomment}[1]{\color{grau}{// #1}}

% Zeilenabstand einstellen %
\renewcommand{\baselinestretch}{1.25}
% Floating-Umgebungen anpassen %
\renewcommand{\topfraction}{0.9}
\renewcommand{\bottomfraction}{0.8}
% Abkuerzungen richtig formatieren %
\usepackage{xspace}
\newcommand{\vgl}{vgl.\@\xspace} 
\newcommand{\zB}{z.\nolinebreak[4]\hspace{0.125em}\nolinebreak[4]B.\@\xspace}
\newcommand{\bzw}{bzw.\@\xspace}
\newcommand{\dahe}{d.\nolinebreak[4]\hspace{0.125em}h.\nolinebreak[4]\@\xspace}
\newcommand{\etc}{etc.\@\xspace}
\newcommand{\evtl}{evtl.\@\xspace}
\newcommand{\ggf}{ggf.\@\xspace}
\newcommand{\bzgl}{bzgl.\@\xspace}
\newcommand{\so}{s.\nolinebreak[4]\hspace{0.125em}\nolinebreak[4]o.\@\xspace}
\newcommand{\iA}{i.\nolinebreak[4]\hspace{0.125em}\nolinebreak[4]A.\@\xspace}
\newcommand{\sa}{s.\nolinebreak[4]\hspace{0.125em}\nolinebreak[4]a.\@\xspace}
\newcommand{\su}{s.\nolinebreak[4]\hspace{0.125em}\nolinebreak[4]u.\@\xspace}
\newcommand{\ua}{u.\nolinebreak[4]\hspace{0.125em}\nolinebreak[4]a.\@\xspace}
\newcommand{\og}{o.\nolinebreak[4]\hspace{0.125em}\nolinebreak[4]g.\@\xspace}
\newcommand{\oBdA}{o.\nolinebreak[4]\hspace{0.125em}\nolinebreak[4]B.\nolinebreak[4]\hspace{0.125em}d.\nolinebreak[4]\hspace{0.125em}A.\@\xspace}
\newcommand{\OBdA}{O.\nolinebreak[4]\hspace{0.125em}\nolinebreak[4]B.\nolinebreak[4]\hspace{0.125em}d.\nolinebreak[4]\hspace{0.125em}A.\@\xspace}

% Leere Seite ohne Seitennummer, naechste Seite rechts
\newcommand{\blankpage}{
 \clearpage{\pagestyle{empty}\cleardoublepage}
}

% Keine einzelnen Zeilen beim Anfang eines Abschnitts (Schusterjungen)
\clubpenalty = 10000
% Keine einzelnen Zeilen am Ende eines Abschnitts (Hurenkinder)
\widowpenalty = 10000 \displaywidowpenalty = 10000
% EOF
